\documentclass[]{report}
\usepackage[margin=2cm]{geometry}
\usepackage{multicol}
\usepackage{hyperref}
\usepackage{color}
\usepackage[utf8]{inputenc}
\usepackage[russian]{babel}
\usepackage[OT1]{fontenc}



%\usepackage{lipsum}

\usepackage{array, xcolor}
\definecolor{lightgray}{gray}{0.8}
\newcolumntype{L}{>{\raggedleft}p{0.14\textwidth}}
\newcolumntype{R}{p{0.8\textwidth}}
\newcommand\VRule{\color{lightgray}\vrule width 0.5pt}

\setlength{\columnseprule}{0.1pt}
\def\columnseprulecolor{\color{black}}

\author{Виталий Епифанцев}
\begin{document}

\section*{\huge Виталий Епифанцев}
\textit{\large C++ разработчик}


\section*{\Large Персональная информация}

\begin{multicols}{3}

\section*{Контакты}
\hrule\medskip
Phone:\\\smallskip +79620783324\\
E-mail:\\\smallskip vitaliy.epifantsev@yandex.ru

\columnbreak

\section*{Git}
\hrule\medskip
Git:\\\smallskip \href{http://www.github.com/DubKoldun}{github.com/DubKoldun}\\
\\\smallskip 

\columnbreak

\section*{Социальные сети}
\hrule\medskip
VK:\\\smallskip \href{http://www.vk.com/bogg_art}{vk.com/bogg\_art}\\
TG:\\\smallskip \href{http://www.t.me/difficultsense}{t.me/difficultsense}
\columnbreak

\end{multicols}



\section*{\Large Образование}

\begin{tabular}{L!{\VRule}R}
2017 -- настоящее время& Студент университета ИТМО, факультет информационных технологий и программирования. Направление: прикладная математика и информатика.
%2017--2021&BSc in Computer Technologies, Great University, Country.\\ %[5pt]
\end{tabular}

%\section*{Professional Experience}
%\begin{tabular}{L!{\VRule}R}
%2011--today&{\bf Work at company XY.}\\
%&\lipsum[66]\\[5pt]
%2008--2010&{\bf Trainee at company ZY.}\\
%&\lipsum[66]\\
%\end{tabular}

\section*{Ключевые навыки}

\begin{enumerate}

\item {{\large Опыт многопоточного программирования}}
\item {{\large Git (VSC)}}
\item {{\large Объектно-Ориентированное программирование и соответствующие принципы}}

\end{enumerate}

\section*{Языки программирования и соответствующие фреймворки}

\begin{enumerate}

\item {
	\textsc{C++}
	\begin{itemize}
		\item {QT Creator}
		\item {C++11/C++14/C++17 стандарты}
	\end{itemize}
}

\item {
	\textsc{Kotlin}
	\begin{itemize}
		\item {Android Studio}
	\end{itemize}
}

\item {\textsc{Perl}
	\begin{itemize}
		\item {Regular expression}
	\end{itemize}
}

\item {
	\textsc{Latex}
	\begin{itemize}
		\item {также Latex вставки для Markdown}   
	\end{itemize}	   		
}

\item {\textsc{Shell}}

\item {\textsc{Linux}}

\item {\textsc{SQL}}

\end{enumerate}


\section*{\Large Языки}
\begin{tabular}{L!{\VRule}R}
Russian&Native\\
{\bf English}&{\bf Intermediate}
\end{tabular}

\end{document}